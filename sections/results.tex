%% LaTeX2e class for student theses
%% sections/conclusion.tex
%% 
%% Karlsruhe Institute of Technology
%% Institute for Program Structures and Data Organization
%% Chair for Software Design and Quality (SDQ)
%%
%% Dr.-Ing. Erik Burger
%% burger@kit.edu
%%
%% Version 1.3.6, 2022-09-28

\chapter{Ergebnisse}
\label{ch:Results}


\begin{figure}
    \todo{WIP, text, vllt mehrere Diagramme}
    \centering
    \begin{tikzpicture}
        \begin{axis}[
            ymajorgrids=true,
            xmin=0, xmax=100,
            grid style=dashed,
            width=0.9\textwidth,
            ylabel={Branchenabdeckung [\%]},
            xlabel={Anzahl der generierten Benutzerinteraktionen [1]},
            no markers,
            legend pos=south east,
            %xtick=data,
            ]
            \addplot[color=green] table [x=x,y=average,col sep=comma] {experimental_data/monkey/results.csv};
            %\addplot+[forget plot,name path=monkey_top,color=green!70] table [x=x,y=above,col sep=comma] {experimental_data/monkey/results.csv};
            %\addplot+[forget plot,name path=monkey_bottom,color=green!70] table [x=x,y=below,col sep=comma] {experimental_data/monkey/results.csv};
            %\addplot+[forget plot,green!50,fill opacity=0.5] fill between[of=monkey_top and monkey_bottom];
            \addlegendentry{Monkey-Testing, n=10}

            \addplot[color=black] table [x=x,y=average,col sep=comma] {experimental_data/gpt4-base/results.csv};
            \addlegendentry{GPT-4 Basiseingabe, n=1}

            \addplot[color=blue] table [x=x,y=average,col sep=comma] {experimental_data/gpt4-chained/results.csv};
            \addlegendentry{GPT-4 Verkettung, n=2}
            %\addplot+[forget plot,name path=gpt4_chain_top,color=blue!70] table [x=x,y=above,col sep=comma] {experimental_data/gpt4-chained/results.csv};
            %\addplot+[forget plot,name path=gpt4_chain_bottom,color=blue!70] table [x=x,y=below,col sep=comma] {experimental_data/gpt4-chained/results.csv};
            %\addplot+[forget plot,blue!50,fill opacity=0.5] fill between[of=gpt4_chain_top and gpt4_chain_bottom];

            \addplot[color=orange] table [x=x,y=average,col sep=comma] {experimental_data/gpt4-describe/results.csv};
            \addlegendentry{GPT-4 Beschreibung, n=3}
            %\addplot+[forget plot,name path=gpt4_describe_top,color=orange!70] table [x=x,y=above,col sep=comma] {experimental_data/gpt4-describe/results.csv};
            %\addplot+[forget plot,name path=gpt4_describe_bottom,color=orange!70] table [x=x,y=below,col sep=comma] {experimental_data/gpt4-describe/results.csv};
            %\addplot+[forget plot,orange!50,fill opacity=0.5] fill between[of=gpt4_describe_top and gpt4_describe_bottom];

            \addplot[color=purple] table [x=x,y=average,col sep=comma] {experimental_data/gpt4-explain/results.csv};
            \addlegendentry{GPT-4 Erklärung, n=5}
            %\addplot+[forget plot,name path=gpt4_explain_top,color=purple!70] table [x=x,y=above,col sep=comma] {experimental_data/gpt4-explain/results.csv};
            %\addplot+[forget plot,name path=gpt4_explain_bottom,color=purple!70] table [x=x,y=below,col sep=comma] {experimental_data/gpt4-explain/results.csv};
            %\addplot+[forget plot,purple!50,fill opacity=0.5] fill between[of=gpt4_explain_top and gpt4_explain_bottom];

            \addplot[color=red] table [x=x,y=average,col sep=comma] {experimental_data/gpt4-goals/results.csv};
            \addlegendentry{GPT-4 Ziele, n=4}

            \addplot[color=gray] table [x=x,y=average,col sep=comma] {experimental_data/gpt4-chain-of-thought/results.csv};
            \addlegendentry{GPT-4 Chain of Thought, n=4}
        \end{axis}
    \end{tikzpicture}
    
\end{figure}

\todo{deutsch, keine Orakel}

As we know of no other approaches to doing scripted testing using only natural language test descriptions, comparing it to other works is hard.
As such validation of this part of the work will center around its feasibility.
The new system's ability to navigate the AUT will be compared to monkey testing.
The test oracle part of the system will be validated using mutation testing and its results will be captured in a confusion matrix.

\section{Vergleich zu Monkey Testing}

Monkey testing is the approach of taking random user actions to find failing application states.
We will measure if the system can successfully navigate the software into the state that shall be tested for each test case and count the number of user interactions it simulates.
That amount of user interaction will be compared to the number of user interactions performed by a monkey tester to navigate to the same state.
The system must at least use fewer user interactions as a monkey tester.
