%% LaTeX2e class for student theses
%% sections/abstract_de.tex
%% 
%% Karlsruhe Institute of Technology
%% Institute for Program Structures and Data Organization
%% Chair for Software Design and Quality (SDQ)
%%
%% Dr.-Ing. Erik Burger
%% burger@kit.edu
%%
%% Version 1.3.6, 2022-09-28

\Abstract

Die Automatisierung von Testprozessen ist ein wichtiger Schritt zur Verbesserung der Softwarequalität und zur Minimierung von Fehlern.
Automatisierte Tests können Zeit und Kosten sparen, indem sie eine schnelle und effiziente Identifizierung und Behebung von Fehlern ermöglichen.
Ein wichtiger Aspekt der Softwarequalität ist die Benutzeroberfläche, die für den Benutzer die wichtigste Schnittstelle zur Anwendung ist.
Diese kann durch explorative Tests, die die Anwendungszustände erkunden, getestet werden.
Das effiziente Erkunden der Anwendungszustände ist jedoch schwierig, da die Anzahl der Pfade durch die Anwendung exponentiell mit der Anzahl der Interaktion steigt.
Ein vielversprechender Ansatz ist die Verwendung von großen Sprachmodellen (LLM) zur Generierung von Benutzeraktionen.
Diese Arbeit testet dies in einem realistischen Szenario und untersucht, wie effektiv das LLM darin ist, Zustände zu erreichen, die mit anderen Methoden schwer zu erreichen sind.