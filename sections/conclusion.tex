\chapter{Fazit}
\label{ch:conclusion}

In dieser Arbeit wurde untersucht, ob Large Language Models (LLMs) für das explorative Testen von Webseiten eingesetzt werden können.
Dazu wurde eine Webshop-Webseite entwickelt und ein Testskript geschrieben, das die Webseite mit verschiedenen Methoden testet und die Branchenabdeckung misst.
Es wurden GPT-3.5 und GPT-4 mit verschiedenen Eingaben und Monkey-Testing verglichen.
GPT-3 erreichte dabei keine besseren Ergebnisse als Monkey-Testing und erzeugte teilweise unmögliche Benutzerinteraktionen.
GPT-4 erreichte mit der gleichen Anzahl an Interaktionen mehr Zustände und somit eine höhere Branchenabdeckung als Monkey-Testing.
Jedoch hatten sowohl GPT-4 als auch Monkey-Testing Probleme mit der Validierungslogik des Webshops, Monkey-Testing konnte keine validen Eingaben generieren und GPT-4 hat die invaliden Eingaben nicht ausreichend getestet.
